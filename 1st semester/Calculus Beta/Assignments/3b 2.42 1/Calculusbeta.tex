
\documentclass[12pt,a4paper]{article}
\title{Calculus $\beta$} 
\author{Rasmus Crolly}
\date{\today}

\usepackage{graphicx}
\usepackage{pgfplots}
\usepackage{amsmath}
\usepackage{amssymb}
\usepackage{textcomp}

\usepackage[
backend=biber,
style=alphabetic,
sorting=ynt
]{biblatex}
\addbibresource{sources.bib}

\usepackage{siunitx}
\usepackage[cm]{fullpage}
\usepackage{listings}

\pgfplotsset{compat=1.16}
\usetikzlibrary{external}
\tikzexternalize[prefix=tikz/]

\newcommand{\R}{\mathbb{R}}

\setlength{\parindent}{0pt}

\usepackage{hyperref}
\hypersetup{
colorlinks=true,
linkcolor=blue,
filecolor=magenta,
urlcolor=cyan,
}				
\urlstyle{same}

% figure support
\usepackage{import}
\usepackage{xifthen}
\pdfminorversion=7
\usepackage{pdfpages}
\usepackage{transparent}
\newcommand{\incfig}[1]{%
	\def\svgwidth{\columnwidth}
	\import{./figures/}{#1.pdf_tex}
}

\pdfsuppresswarningpagegroup=1

\begin{document}
\maketitle

Opskriv det generelle differentiale for de følgende funktioner og angiv differentialet i punktet (1,2):
\subsection{a}


\[
	f\left( x,y \right)=xy^2 
.\] 
First off the partial derivatives are determined
\[
\frac{\partial f}{\partial x}=y^2
.\] 

\[
\frac{\partial f}{\partial y}=2xy 
.\] 

The general differential is therefore \[
	df=\left( y^2 \right) dx\left( 2xy \right) dy
.\] 


the  differential in (1,2) is therefore

\[
df=2^2dx+2\cdot 1\cdot 2 dy=4dx+4dy
.\] 


\subsection{b}
Given the function \[ x^{2} \sin{\left(\pi y^{2} \right)} .\]the partial derivative for a given variable is determined by differentiating the original function with the respective variable and keeping all other variables constant 
$f_x$ is therefore
\[ f_x = \frac{\partial}{\partial x} = 2 x \sin{\left(\pi y^{2} \right)} .\]
Differentiating $f(x,y)$ for $y$ requires differentiating the function $ \sin{\left(\pi y^{2} \right)} .$
Since this is a composite function the chain rule is used
\[ (f(g(x)))'=f'(g(x))\cdot g'(x) .\]
the following functions are chosen as f and g respectively
\[ f = \sin{\left(y \right)} ,\; g = \pi y^{2} .\]
this means f' and g' become
\[ f' = \cos{\left(y \right)} ,\; g' = 2 \pi y .\]
these can now all be inserted into the chain rule
\[ 2 \pi y \cos{\left(\pi y^{2} \right)} .\]
then multiply with the constant $ x^{2} $ to obtain the full expression for $f_y$
\[ f_y = \frac{\partial}{\partial y} = 2 \pi x^{2} y \cos{\left(\pi y^{2} \right)} .\]
Finally inputting the original equation in a CAS program and finding the partially derived for y returns
\[ f_y = \frac{\partial}{\partial y} = 2 \pi x^{2} y \cos{\left(\pi y^{2} \right)} .\]
which suggests the method used was correct
the generel differential can now be found by insering $f_x$ and $f_y$
\[ 2 dx x \sin{\left(\pi y^{2} \right)} + 2 \pi dy x^{2} y \cos{\left(\pi y^{2} \right)} .\]
and inserting the point
\[ 4 \pi dy .\]
the differential in (1,2) is therefore
\[ 4 \pi dy .\]


\end{document}
