
\documentclass[12pt,a4paper]{article}
\title{Mechanics and Thermodynamics}
\author{Anon}
\date{\today}

\usepackage[utf8]{inputenc}
\usepackage{graphicx}
\usepackage{pgfplots}
\usepackage{amsmath}
\usepackage{amssymb}
\usepackage[T1]{fontenc}
\usepackage{textcomp}

\usepackage[
backend=biber,
style=alphabetic,
sorting=ynt
]{biblatex}
\addbibresource{sources.bib}

\usepackage{siunitx}
\usepackage[cm]{fullpage}
\usepackage{listings}

\pgfplotsset{compat=1.16}
\usetikzlibrary{external}
\tikzexternalize[prefix=tikz/]

\newcommand{\R}{\mathbb{R}}

\setlength{\parindent}{0pt}

\usepackage{hyperref}
\hypersetup{
colorlinks=true,
linkcolor=blue,
filecolor=magenta,
urlcolor=cyan,
}				
\urlstyle{same}

% figure support
\usepackage{import}
\usepackage{xifthen}
\pdfminorversion=7
\usepackage{pdfpages}
\usepackage{transparent}
\newcommand{\incfig}[1]{%
	\def\svgwidth{\columnwidth}
	\import{./figures/}{#1.pdf_tex}
}

\pdfsuppresswarningpagegroup=1

\begin{document}
\maketitle
\section{Objects on a ring}

\subsection{1}

\subsubsection{i}

In the initial stage the heavier object has a kinetic energy of 0 since it is at rest.

The potential energy in the initial position consists of two parts

\begin{itemize}
	\item potential energy due to gravity
	\item potential energy due to the pull of the spring
\end{itemize}


Potential energy due to gravity is calculated by \[
	E_{grav}=mgh
.\] 
where $m$ is the mass of the object,  $g$ is the gravitational pull and $h$ is the height

Gravitational potential energy can therefore be expressed as \[
3m\cdot g\cdot 2R = 6mgR
.\] 

The potential energy stored in a spring is expressed as \[
U_{el} = \frac{1}{2}kx^2
.\] 

Where the $x$-axis has the same direction as the spring. The lenght of  $x$ can be found by using pythagoras theorem \[
a^{2}+b^{2}=c^{2}
.\] 
$c$ and therefore $x$ is thus \[
\sqrt{R^{2}+2R^{2}} 
.\] 

resulting in the final expression for $U_{el}$ \[
	U_{el}=\frac{1}{2}k\left( R^2+2R^2 \right) 
.\] 

By adding the expressions for $U_{el}$ and $E_{grav}$ the expression for the total potential energy in state can be found \[
	E_{i}=6mgR+\frac{1}{2}k\left( R^{2}+2R^{2} \right) 
.\] 

\subsubsection{ii}

in the second state 
\[
	W_{el}=\frac{1}{2}k\left( R^2+2R^2 \right) -\frac{1}{2}kR^2
.\] 
\[
	W_{grav}=-3mg\left( R-2R \right) 
.\] 

total work done 

\[
	W_{tot,ii}=(\frac{1}{2}k3R^2 -\frac{1}{2}kR^2)*3mgR=3kmgR^3
.\] 

kinetic energy before and after

\[
K_{1}+
.\] 



\end{document}
